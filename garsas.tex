\chapter{Garsas}

  % TODO: daugiau info, daugiau info!
  
  Garsui Arch Linux sistemoje apdoroti yra naudojamas ``Advanced Linux
  Sound Architekture'' ( taip pat dar žinomas kaip ALSA ). Jis yra
  Linux sisteminio branduolio ( kernel ) komponentas, kuris yra
  skirtas pakeisti originalų ``Open Sound System'' ( taip pat dar
  žinoma kaip OSS ). Be garso plokščių atpažinimo, ALSA taip pat
  atlieka pridėtinės programinės įrangos vaidmenį, t.y. ji suteikia
  programuotojams pagrindą aukštesnio lygio API, kuris tiesiogiai
  sąveikauja su sisteminio branduolio ( kernel ) tvarkyklėmis.

  \paragraph{Pastaba} Alsa yra įtraukta į pagrindinį Arch Linux
  sisteminio branduolio paketą ir udev automatiškai atpažins
  kompiuterio garso plokštę, bei užkraus reikiamą tvarkyklę/modulį į
  sisteminį branduolį. Taip pat, garsas sistemoje jau turi veikti,
  tiesiog pagal numatytus nustatymus visi kanalai yra užslopinti.

  Garso kanalams atidaryti reikia \textsl{alsamixer} komandos, kuri
  yra paketo \textsl{alsa-utils} dalis. Diegimui įveskite sekančią
  komandą:

  \begin{verbatim}
# pacman -S alsa-utils
  \end{verbatim}

  Taip pat rekomenduojama diegti \textsl{alsa-oss} paketą, kuris
  suteiks programoms, naudojančioms OSS tiltą prie ALSA:

  \begin{verbatim}
# pacman -S alsa-oss
  \end{verbatim}

  Jeigu ne-root vartotojas dar nėra įtrauktas į \textsl{audio}
  vartotojų grupę, tai galima atlikti su \textsl{gpasswd} komanda:

  \begin{verbatim}
# gpasswd -a vartotojo-vardas audio
  \end{verbatim}

  Grupės pridėjimas yra įvykdytas, tačiau dabartinėje vartotojo
  sesijoje, paprastas vartotojas vis dar nėra prijungtas prie
  \textsl{audio} grupės. Tam, kad nauji nustatymai įsigaliotų, reikia
  išsiregistruoti iš sistemos ir vėl prisiregistruoti prie sistemos.

  Kaip ne-root, paprastas vartotojas įvykdykite \textsl{alsamixer}:

  \begin{verbatim}
# alsamixer
  \end{verbatim}

  Norint atidaryti tam tikrą kanalą, pirmą reikia jį įjungti,
  paspaudžiant ``M'' raidę. Garso reguliavimas atliekamas su krypties
  klavišais į viršų ir į apačią. Tam, kad pereiti prie kito kanalo,
  paspauskite krypties klavišus į dešinę arba į kairę.

  Dažniausiai reikia įjungti ir duoti maksimumą ``Master'' ir jo
  analogijoms, pavyzdžiui ``Master Mono'', ``Headphone'' ir kiti,
  kurie yra prieš ``PCM''. Su ``PCM'' galima atlikti pagrindinį
  sistemos garso reguliavimą.

    \subsection{Garso testavimas}

      Garso nustatymus patikrinti galima su \textsl{aplay} komanda:

      \begin{verbatim}
# aplay /usr/share/sounds/alsa/Front_Central.wav
      \end{verbatim}

      Turėtumėt išgirsti malonų moters garsą, kuris sako ``Front,
      center''.

    \subsection{Garso nustatymų išsaugojimas}

      Persijunkite prie root vartotojo ir išsaugokite visus garso
      nustatymus, naudojant \textsl{alsactl} komanda:

      \begin{verbatim}
# alsactl -f /var/lib/alsa/asound.state store
      \end{verbatim}

      Šita komanda sukurs naują bylą, pavadinimu
      \textsl{asound.state}, kurioje bus išsaugoti visi ALSA
      reikalingi konfigūraciniai duomenys.

      Galiausiai, reikia įtraukti alsa daemon'ą į DEAMONS
      \textsl{/etc/rc.conf} masyvą, kad kiekvieną kartą, paleidžiant
      sistemą, garso nustatymai atsistatytų:

      \begin{verbatim}
DAEMONS=(syslog-ng @network crond alsa)
      \end{verbatim}
