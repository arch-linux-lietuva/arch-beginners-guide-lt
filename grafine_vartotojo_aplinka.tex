\chapter{Grafinė vartotojo aplinka}

  \section{Žingsnis 1: X diegimas}

    \textbf{X} 11 versijos langų sistema ( taip pat žinoma kaip
    \textbf{X11} arba tiesiog \textbf{X} ) yra tinklo ir atvaizdavimo
    prievadas, kuris suteikia langų aplinką taškiniuose
    vaizduokliuose. Jis turi savyje visus reikiamus įrankius grafinės
    aplinkos ( GUI ) kūrimui.

    \textbf{X} suteikia pagrindinę bazę, GUI kūrime: vietą pelytės
    sąsajai, klaviatūrai ar kitiems įvestiems įrenginiams, operuojant
    su langais darbastalyje. \textbf{X} nesuteikia vartotojo aplinkos
    įgaliojimus kiekvienam langui. Tuo turi pasirūpinti pačios
    programos. \textbf{X} nesuteikia grafinio grožio, kadangi tam
    reikės dar vieno sluoksnio. \textbf{X} yra taip pavadintas,
    kadangi jis kilo iš \textbf{W} langų sistemos, kurią kūrė
    Standford'o Universitete.

    \subsection{A: Xorg diegimas}

      Dabar mes diegsime Xorg, naudodami pacman. Tai yra pirmas
      žingsnis grafinės aplinkos kūrime:

      \begin{verbatim}
# pacman -S xorg
      \end{verbatim}

      3D palaikymui, reikia diegti mesa:

      \begin{verbatim}
# pacman -S mesa
      \end{verbatim}

      3D įrankiai, tokie kaip \textsl{glxgears} arba \textsl{glxinfo}
      yra \textbf{mesa-demos} pakete:

      \begin{verbatim}
# pacman -S mesa-demos
      \end{verbatim}

    \subsection{B: Vaizdo tvarkyklės diegimas}

      Toliau seka vaizdo tvarkyklės diegimas. Šiame žingsnyje
      reikalaujama žinoti koks yra kompiuterio lustų rinkinys. Jeigu
      tai ir nėra žinoma, tai nesunku sužinoti su \textsl{lspci}
      programa:

      \begin{verbatim}
# lspci
      \end{verbatim}

      \paragraph{Pastaba} \textbf{lspci} programa parodo visus lustus,
      kurie yra prijungti prie kompiuterio per PCI jungtį.

      \paragraph{Pastaba} Jeigu reikiamos tvarkyklės nepavyksta rasti,
      tuomet iš bėdos galima naudoti ir \textbf{vesa} vaizdo
      tvarkyklę. Tačiau ji siūlo tik lėtą 2D aplinką. 

      Jeigu reikia diegti atviro kodo tvarkykles, įvykdykite:

      \begin{verbatim}
# pacman -Ss xf86-video | less
      \end{verbatim}

      Arba naudokite šį atrivo kodo tvarkyklių sąrašą

      \begin{itemize}
        \item \textbf{xf86-video-apm} - Alliance ProMotion vaizdo
          tvarkyklė
        \item \textbf{xf86-video-ark} - ark vaizdo tvarkyklė
        \item \textbf{xf86-video-ast} - ASPEED AST vaizdo tvarkyklė
        \item \textbf{xf86-video-ati} - ATI(AMD) radeon vaizdo
          tvarkyklė
          \begin{itemize}
            \item \textbf{xf86-video-r128} - ATI(AMD) vaizdo tvarkyklė
              ati Rage128 vaizdo kortai
            \item \textbf{xf86-video-mach64} - ATI(AMD) vaizdo
              tvarkyklė mach64 vaizdo kortai
          \end{itemize}
        \item \textbf{xf86-video-chips} - Chips and Technologies vaizdo tvarkyklė
        \item \textbf{xf86-video-cirrus} - Cirrus Logic vaizdo tvarkyklė
        \item \textbf{xf86-video-dummy} - fiktyvi vaizdo tvarkyklė
        \item \textbf{xf86-video-fbdev} - framebuffer vaizdo tvarkyklė
        \item \textbf{xf86-video-glint} - GLINT/Permedia vaizdo tvarkyklė
        \item \textbf{xf86-video-i128} - Number 0 i128 vaizdo tvarkyklė
        \item \textbf{xf86-video-i740} - Intel i740 vaizdo tvarkyklė
        \item \textbf{xf86-video-intel} - Intel i810/i830/i915/945G/G965+ vaizdo tvarkyklė
        \item \textbf{xf86-video-mga} - Mga ( Matrox Graphics Adapter ) vaizdo tvarkyklė
        \item \textbf{xf86-video-neomagic} - neomagic vaizdo tvarkyklė
        \item \textbf{xf86-video-nv} - NVidia nv vaizdo tvarkyklė
        \item \textbf{xf86-video-nouveau} - Open Source 3D
          acceleration tvarkyklė nVidia vaizdo kortoms
        \item \textbf{xf86-video-openchrome} - VIA/S3G UniChrome,
          UniChrome Pro ir Chrome9 vaizdo tvarkyklė
        \item \textbf{xf86-video-rendition} - Rendition vaizdo tvarkyklė
        \item \textbf{xf86-video-s3} - S3 vaizdo tvarkyklė
        \item \textbf{xf86-video-s3virge} - S3 Virge vaizdo tvarkyklė
        \item \textbf{xf86-video-savage} - Savage vaizdo tvarkyklė
        \item \textbf{xf86-video-siliconmotion} - Siliconmotion vaizdo tvarkyklė
        \item \textbf{xf86-video-sis} - SiS vaizdo tvarkyklė
        \item \textbf{xf86-video-sisusb} - SiS USB vaizdo tvarkyklė
        \item \textbf{xf86-video-tdfx} - Tdfx vaizdo tvarkyklė
        \item \textbf{xf86-video-trident} - Trident vaizdo tvarkyklė
        \item \textbf{xf86-video-tseng} - Tseng vaizdo tvarkyklė
        \item \textbf{xf86-video-unichrome} - VIA S3 Unichrome vaizdo tvarkyklė
        \item \textbf{xf86-video-v4l} - V4l vaizdo tvarkyklė
        \item \textbf{xf86-video-vesa} - Vesa vaizdo tvarkyklė
        \item \textbf{xf86-video-vmware} - Vmware vaizdo tvarkyklė
        \item \textbf{xf86-video-voodoo} - Doovoo vaizdo tvarkyklė
        \item \textbf{xf86-video-xgi} - XGI vaizdo tvarkyklė
        \item \textbf{xf86-video-xgixp} - XGIXP vaizdo tvarkyklė
      \end{itemize}

      Norint įdiegti vaizdo tvarkyklę, naudokite pacman, pavyzdžiui:

      \begin{verbatim}
# pacman -S xf86-video-vesa
      \end{verbatim}

      \subsubsection{NVIDIA vaizdo plokštės}

        NVIDIA vartotojai turi tris pasirinkimus ( nepaisant vesa
        tvarkyklės ):
        
        \begin{itemize}
          \item Atviro kodo nouveau tvarkyklė, kuri siūlo 2D
            greitinimą ir eksperimentinį 3D palaikymą, kuris gerai
            tinka komponavimo ( compositing ) reikmėms, tačiau jis
            nepalaiko tokių dalykų kaip energijos taupymas.
          \item Atviro kodo ( tačiau klaidinantis ) nv tvarkyklė,
            kuri yra labai lėta ir siūlo tik 2D palaikymą.
          \item Atitinkamas nvidia tvarkykles, kurios siūlo gerą 3D
            palaikymą, bei energijos taupymą. Jeigu yra planuose
            naudoti atitinkamą nvidia tvarkyklę, pirmiausiai reiktų
            pradėti nuo nouveau tvarkyklės, o tik tuomet persijungti
            prie nvidia tvarkyklės, kadangi pastaroji veikia iškarto,
            o nvidia reikalaus konfigūracijos ir gali kilti tam tikrų
            problemų, kurias teks spręsti.
        \end{itemize}

        Atviro kodo nouveau yra patenkinama daugumai vartotojų ir yra
        rekomenduojama:

        \begin{verbatim}
# pacman -S xf86-vide-nouveau
        \end{verbatim}

        3D palaikymui ( eksperimentinis ):
        
        \begin{verbatim}
# pacman -S nouveau-dri
        \end{verbatim}

        Sukurkite naują \textsl{/etc/X11/xorg.conf.d/20-nouveau.conf}
        bylą ir pridėkite sekantį turinį:

        \begin{verbatim}
Section "Device"
  Identifier "n"
  Driver "nouveau"
EndSection
        \end{verbatim}

        Reikia įsitikinti, jog nouveau tvarkyklė bus pakrauta. Xorg
        dar nėra pakankamai protingas, kad tai atliktų automatiškai.

      \subsubsection{ATI vaizdo plokštės}

        ATI vartotojai turi du tvarkyklių pasirinkimus ( nepaisant
        vesa tvarkyklės ):

        \begin{itemize}
          \item Atviro kodo \textbf{radeon} tvarkyklė, kuri yra paketo
            \textbf{xf86-video-ati} dalis. Ji pilnai palaiko visas ATI
            kortas iki X1950 ( naujausias R500 lustų rinkinys
            ). Kortos iki 9200 yra pilnai palaikomos, veikia
            stabiliai ir suteikia 2D ir 3D greitinimą. Kortos nuo 9500
            iki HD4000 suteikia pilną 2D greitinimą, ir iš dalies
            stabilų 3D greitinimą, tačiau jai trūksta kai kurių
            specifinių dalykų, kurias siūlo atitinkamų plokščių eilė:
            energijos taupymas yra vis dar kuriamas ir yra
            išplėstinėje stadijoje, tačiau jis visiškai atskiras nuo
            catalyst. HD5000 palaikymas yra dar kūrimo
            stadijoje. Palaiko KMS ir HDMI su garso išėjimu nuo 2.6.33
            branduolio versijos.
          \item Atitinkama \textbf{fglrx} tvarkyklė įtraukta į
            catalyst paketą yra pasiekiama AUR saugykloje. Ji palaiko
            tik naujausius įrenginius ( HD2xxx ir naujesnius
            ). Ankščiau paketas buvo įtraukas į \textsl{extra}
            saugyklą, tačiau 2009 kovą jo palaikymas buvo nutrauktas
            dėl kokybės ir greičio prastumo.
        \end{itemize}

        Atviro kodo tvarkyklė yra rekomenduojamas
        pasirinkimas:

        \begin{verbatim}
# pacman -S xf86-video-ati
        \end{verbatim}
    
    \subsection{C: Įvesties ( input ) tvarkyklių diegimas}

      Udev turi atlikti kompiuterio aparatinės įrangos atpažinimą, o
      evdev ( xf86-input-evdev ) yra modernus, savaiminio atpažinimo
      įvesties tvarkyklė beveik visiems įrenginiams, tad daugumoje
      atveju, įvesties tvarkyklių įdiegti nereikia. Šiame žingsnyje,
      evdev jau yra įdiegtas į kompiuterį kaip Xorg priklausomybė.

      Jeigu evdev nepalaiko įrenginio, tuomet reikia diegti tvarkyklę,
      kuri yra xorg-input-drivers grupėje.

      Visiškam palaikomų įvesties tvarkyklių sąrašui įvykdykite
      sekančią komandą:

      \begin{verbatim}
# pacman -Ss xf86-input | less
      \end{verbatim}

      \paragraph{Pastaba} Tereikia tik xf86-input-keyboard arba
      xf86-input-mouse, jeigu planuojate nenaudoti savaiminio
      atpažinimo.

      Nešiojamų kompiuterių savininkams ( arba vartotojams su lietimu
      jautriais ekranais ) dar reikia diegti \textsl{synaptics}
      paketą, kuris leis \textbf{X} sukonfigūruoti jutikliniam
      kilimėliui arba jutikliniam ekranui:

      \begin{verbatim}
# pacman -S xf86-input-synaptics
      \end{verbatim}

  \section{Žingsnis 2: X konfigūravimas ( nebūtina )}

    X.org X serveris turi auto konfigūracijos galimybę. Taip pat, X
    gali funkcionuoti ir visiškai be vartotojo nurodytos
    konfigūracijos.

    Sąrašas konfigūracinių bylų yra \textsl{/etc/X11}
    direktorijoje. Šitame kataloge yra \textsl{xorg.conf.d} katalogas,
    kuriame saugomi visos numatytos konfigūracinės bylos X
    serveriui. Tai yra Arch Linux distribucijos sukurtos numatytos
    konfigūracinės bylos, bazė vartotojų redagavimui. Taip pat lieka
    galimybė tiesiog sukurti paprastą \textsl{/etc/X11/xorg.conf},
    kuris bus skaitomas paskutinis, po visų \textbf{xorg.conf.d}
    kataloge esančių konfigūracinių bylų.

    \subsection{Ne-US klaviatūra}

      % TODO:
      % Papasakoti kaip galima padaryti layout switch'a

      Jeigu naudosite ne standartinę US tipo klaviatūrą, reikia
      nustatyti klaviatūros išdėstymą
      \textsl{/etc/X11/xorg.conf.d/10-evdev.conf} byloje:

      \begin{verbatim}
Section "InputClass"
    Identifier "evdev keyboard catchall"
    MatchIsKeyboard "on"
    MatchDevicePath "/dev/input/event*"
    Driver "evdev"
    Option "XkbLayout" "lt"
EndSection
      \end{verbatim}

  \section{Žingsnis 3: Bazinės grafinės aplinkos paleidimas}

    Šitam žingsnyje paaiškinsime kaip paleisti bazinę grafinę aplinką,
    kuri yra įtrauka į Xorg paketą. Tokia grafinė aplinka naudoja
    paprasčiausią langų tvarkyklę, twm. Numatyta X aplinka yra labai
    primityvi, tačiau greitu metu diegsime geresnę langų tvarkykle.

    Kadangi Xorg diegimas jau pabaigtas, paprasto vartotojo namų
    kataloge turi būti konfigūracinė byla \textsl{.xinitrc}. Šiame
    žingsnyje bylą reikia ištrinti. Jeigu bylos nepašalinsite, vietoj
    tikėtinos langų tvarkyklės bus rodomas tik juodas ekranas, o
    \textsl{Xorg.0.log} nieko nerodys. Paprasčiausiai pašalinkite šitą
    bylą, kad pasiekti numatytą langų tvarkyklę:

    \begin{verbatim}
# rm ~/.xinitrc
    \end{verbatim}

    \subsection{Pranešimų magistralė}

      Dbus yra reikalaujamas daugelių programų stabiliam
      veikimui. Jeigu žinote, jog jis reikalingas nebus - praleiskite
      sekančią sekciją.

      \subsubsection{Diegimas}

        Dbus diegiamas su pacman pagalba:

        \begin{verbatim}
# pacman -S dbus
        \end{verbatim}

      \subsubsection{Deamon paleidimas}

      Reikia pridėti \textbf{dbus} į DEAMONS masyvą
      \textsl{/etc/rc.conf} byloje:

      \begin{verbatim}
DEAMONS=(syslog-ng dbus @network crond alsa)
      \end{verbatim}
      
      Jeigu norite paleisti dbus neperkraunant sistemos:

      \begin{verbatim}
# /etc/rc.d/dbus start
      \end{verbatim}

    \subsection{X paleidimas}

      Galiausiai, galime leisti Xorg:

      \begin{verbatim}
# startx
      \end{verbatim}

      Arba:

      \begin{verbatim}
# xinit -- /usr/bin/X -nolisten tcp
      \end{verbatim}

      Jeigu monitorius rodo tik juodą ekraną, tai dar nėra
      bėda. Galite persijungti prie kitos konsolės ( pavyzdžiui
      CTRL+ALT+F2 ) ir prisijungti kaip root.

      Galite pabandyti užmušti X serverį su \textsl{/usr/bin/pkill}:

      \begin{verbatim}
# pkill X
      \end{verbatim}

      \paragraph{Pastaba} X raidė turi būti didžioji.

      Jeigu pkill nesuveiks, tai galima aklai perkrauti kompiuterį:

      \begin{verbatim}
# reboot
      \end{verbatim}

      Arba:
      
      \begin{verbatim}
# init 6
      \end{verbatim}

    \subsection{Klaidų atveju}

      Jeigu kyla kažkokia problema, klaidos pranešimo ieškokite
      \textsl{/etc/log/Xorg.0.log} byloje. Ieškokite eilučių, kurios
      prasideda \textsl{EE}, kas reiškia klaida, bei kurios prasideda
      \textsl{WW}, kas reiškia perspėjimas.

      \begin{verbatim}
# grep EE /var/log/Xorg.0.log
      \end{verbatim}

      Klaidos taip pat rodomos konsolėje, iš kurios buvo leidžiamas \textsl{X}.

    \subsection{Reikia pagalbos?}

      Jeigu kyla problemos, kurių išspręsti dar negalite, galite
      kreiptis į forumus:

      \begin{itemize}
        \item http://sls.archlinux.lt - Arch Linux Lietuva forumas
        \item http://bbs.archlinux.org - Oficialus Arch Linux forumas
      \end{itemize}

      Taip pat įsitikinkite, jog turite wgetpaste paketą:

      \begin{verbatim}
# pacman -S wgetpaste
      \end{verbatim}

      Naudodami įrankį, pateikite forume tokių bylų tekstus:

      \begin{itemize}
        \item ~/.xinitrc
        \item /etc/X11/xorg.conf
        \item /var/log/Xorg.0.log
        \item /var/log/Xorg.0.log.old
      \end{itemize}

      Wgetpaste naudoti yra labai paprasta:

      \begin{verbatim}
# wgetpaste /absoliutus/kelias/iki/bylos        
      \end{verbatim}

  \section{Žingsnis 4: Šriftų diegimas}

    Šiuo metu yra gerai išsaugoti savo laiką ir iškarto diegti
    šriftus. Dejavu yra didelės raiškos ir universalių šriftų
    rinkinys:

    \begin{verbatim}
# pacman -S ttf-dejavu
    \end{verbatim}

  \section{Žingsnis 5: Grafinės aplinkos pasirinkimas ir diegimas}

    X langų sistema suteikia pagrindą grafinės vartotojo aplinkos
    sukūrimui ( GUI ). Langų sistema kontroliuoja vietą, kurioje
    atsiranda langai, bei jų išvaizdą. Darbalaukio aplinka ( DE ),
    veikia virš ir su X, suteikdama visiškai funkcinę ir dinamišką
    GUI. Paprastai DE suteikia langų tvarkyklę, ikonas, programėles,
    langus, mygtukų juostas, katalogus, darbalaukio fonus, bei
    pagrindines funkcijas kaip vilkimas.

    Kaip alternatyva DE, galite pasirinkti bet kokią langų tvarkyklę
    ir išsirinkti savo programų rinkinį. 

    \paragraph{Pastaba} Daugumos distribucijų nusprendžia už vartotoją
    kokia darbalaukio aplinka pasirinkti. Arch Linux šitą pasirinkimą
    palieka vartotojui. Darbalaukio aplinkos ar/ir langų tvarkyklės
    pasirinkimas yra asmeninis bei skonio reikalas. Pasirinkite
    geriausiai tinkantį sprendimą pagal esamą poreikį. Rekomenduojama
    pabandyti visas galimas darbalaukio aplinkas ar/ir langų
    tvarkykles ir išsirinkti tą, kuri yra arčiausiai širdies.

    \subsection{Darbastalio aplinka}

      % TODO

    \subsection{Langų tvarkyklės}

      % TODO

  \section{Grafinės aplinkos paleidimo metodai}

    \subsection{A: Rankinis}

      Turbūt geriausias sprendimas yra leisti X rankiniu būdu, negu
      iškarti šokti prie darbastalio. Jeigu tai yra nepriimtina -
      šokite prie sekančio skyriaus ``B: Automatinis''. Yra trys
      galimi rankiniu būdu paleisti X būdai. Pasirinkite tą, kuris yra
      labiausiai tinkamas:

      \subsubsection{Metodas 1: Specifinės GA komandos}

        Kai kuriems vartotojams šis metodas yra paprasčiausias,
        kadangi užtenka tik vienos komandos grafinės aplinkos
        paleidimui. Paprasčiausiai pasinaudokite komanda, kuri
        paleidžia tam tikrą darbalaukio aplinką arba langų
        tvarkyklę. Komandą pradėkite su \textsl{xinit} arba
        \textsl{startx} o toliau nurodykite kokią darbo aplinką
        paleisti:

        \begin{verbatim}
# xinit /usr/bin/gnome-session
        \end{verbatim}

        Arba:

        \begin{verbatim}
# startx /usr/bin/startkde
        \end{verbatim}

        \paragraph{Pastaba} Sekančios komandos baigia savo vykdymą tik
        tuomet, kai bus įvykdytas išregistravimas iš darbalaukio
        aplinkos.

        \paragraph{Pastaba} Naudokite pilną kelią iki darbalaukio
        aplinkos/langų tvarkyklės. Jeigu, pavyzdžiui, paleisite gnome
        kaip \textsl{xinit gnome-session}, gali kilti tam tikrų
        problemų.

        Kaip alternatyvą ilgam komandos rašymui, galima sukurti
        alternatyvų vardą \textsl{/etc/bash.bashrc} byloje:

        \begin{verbatim}
...
alias launch_gnome="xinit /usr/bin/gnome-session"
alias launch_kde="xinit /usr/bin/startkde"
alias launch_openbox="xinit /usr/bin/openbox-session"
...
        \end{verbatim}

        Tokiu atveju, tereiks surinkti \textsl{launch$\_$gnome}
        terminale ir pasileis gnome darbastalio aplinka.

        Dar galima naudoti prisijungimo vadovą, kuriame galima
        pasirinkti kokią darbastalio aplinką paleisti. Jeigu turite
        gnome, tai jo prisijungimo vadovas yra gdm:

        \begin{verbatim}
# gdm
        \end{verbatim}

      \subsubsection{Metodas 2: Naudojant runlevels}

        Pagal numatytus nustatymus, Linux yra keli skirtingi
        runlevels. Arch kraunasi į trečią runlevel. Penktas runlevel
        naudojamas Linux X serverio krovimui. Pakoreguokite savo
        \textsl{/etc/inittab}. Paskutinėje sekcijoje, bylos pabaigoje
        atkomentuokite atitinkamą eilutę savo darbastalio aplinkos
        programai. Jeigu naudojate gnome, tai būtų:

        \begin{verbatim}
# Example lines for starting a login manager
#x:5:respawn:/usr/bin/xdm -nodaemon
x:5:respawn:/usr/sbin/gdm -nodaemon
#x:5:respawn:/usr/bin/kdm -nodaemon
#x:5:respawn:/usr/bin/slim >/dev/null 2>&1
        \end{verbatim}

        Dabar tiesiog paleiskite savo darbastalio programą:

        \begin{verbatim}
# init 5
        \end{verbatim}

      \subsubsection{Metodas 3: Naudojant .xinitrc}

        Šitas metodas reikalauja daugiausiai
        konfigūravimo. Pirmiausiai reikia sukonfigūruoti
        \textsl{~/.xinitrc}. Pagrindinė šios bylos paskirtis yra
        nurodyti X langų serveriui ką daryti per
        \textsl{/usr/bin/startx} arba \textsl{/usr/bin/xinit}
        kiekvieno vartotojo asmeniniam pasirinkimui. Yra begalės
        konfigūracinių ypatybių, kurias galite įrašyti į savo
        \textsl{~/.xinitrc}.

        \textsl{Startx} ar \textsl{xinit } paleis X serverį, bei jo
        klientą. Nuspręsti, kokį klientą paleisti, \textsl{startx}/\textsl{xinit}
        pirmiausiai žiūrės į \textsl{~/.xinit}, kuris yra vartotojo
        namų kataloge. Jeigu sekančios bylos nėra, komandos žiūri į
        numatytą konfigūracinę bylą \textsl{xinit}, kuri yra
        \textsl{/etc/X11/xinit/xinitrc}, kuri sukonfigūruota paleisti
        TWM langų tvarkyklę.

        \begin{itemize}

          \item Persijunkite į paprastą, ne-root vartotoją:

            \begin{verbatim}
# su - vartotojo-vardas
            \end{verbatim}
            

          \item Kaip šabloną, savo konfigūracijoms, galima naudoti
            \textsl{/etc/skel/} kataloge esantį
            \textsl{.xinitrc}. Pavadinimas \textsl{skel} yra žodžio
            \textsl{skeleton} trumpinys. Nukopijuokite bylą iš
            katalogo į savo namų direktoriją:

            \begin{verbatim}
# cp /etc/skel/.xinitrc ~/
            \end{verbatim}

          \item Atsidarykite bylą:

            \begin{verbatim}
# nano ~/.xinitrc
            \end{verbatim}

            Ir atkomentuokite eilutę, kuri nusako naudojama
            darbalaukio aplinkos programą. Jeigu naudosite gnome, tai
            konfigūracinė byla turi atrodyti panašiai:

            \begin{verbatim}
#!/bin/sh
#
# ~/.xinitrc
#
# Executed by startx (run your window manager from here)

exec gnome-session
# exec startkde
# exec startxfce4
# exec wmaker
# exec icewm
# exec blackbox
# exec fluxbox
# exec openbox-session
# ...or the Window Manager of your choice
# exec xterm
            \end{verbatim}

            \paragraph{Pastaba} Įsitikinkite, jog atkomentavote tik
            \textbf{vieną} eilutę.

        \end{itemize}

        Dabar viskas yra paruošta X serverio paleidimui. Paleiskite
        \textbf{X} kaip paprastas, ne-root vartotojas:

        \begin{verbatim}
# startx
        \end{verbatim}
        
        Arba:

        \begin{verbatim}
# xinit
        \end{verbatim}

        Dabar darbalaukio aplinka turėtų sėkmingai pasirodyti
        ekrane. Patikrinkite kaip veikia klaviatūra, pelytė.

        Jeigu kyla problemų su automatiniu įrengimu ( automount ),
        pabandykite savo \textsl{~/.xinitrc} tarp \textsl{exec} ir
        darbalaukio aplinkos komandos įterpti
        \textsl{ck-launch-session}. Pavyzdžiui:

        \begin{verbatim}
exec ck-launch-session gnome-session
        \end{verbatim}

        Tai užtikrins globalių kintamųjų priskyrimą, kai paleisite
        darbalaukio aplinkos programą. Komanda
        \textsl{Ck-launch-session} išvalo \textsl{consolekit}
        sesiją. Consolekit yra programinis pagrindas, kuris yra
        atsakingas už įvairius vartotojus, sesijas. Jis suteikia
        programinei įrangai mechanizmą, kuris leidžia programoms
        pasiekti visus reikiamus kintamuosius. Jis dirba kartu su dbus
        ir kitais įrankiais.

    \subsection{B: Automatinis}

      Daugelis vartotojų gali pasirinkti automatinį grafinės aplinkos
      pakrovimą. Apie tai dar reiks parašyti, kadangi Pradžiamokslių
      vadove to nėra..
