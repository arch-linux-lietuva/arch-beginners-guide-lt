\chapter{Naujo vartotojo sukūrimas}

  Linux yra daugelio naudotojų sistema. Tikrai nereikia naudoti root
  vartotojo kasdieniams darbams. Tai labai bloga praktika ir tai labai
  pavojinga. Root vartotojas yra skirtas tik Linux sistemos
  administravimo darbams. Jis tikrai nėra skirtas programuoti ar
  rašyti kursinius darbus.

  Rekomenduojama sukurti naują, paprastą, ne root tipo vartotoją,
  pasitelkiant \textsl{/usr/sbin/useradd} programą. 

  \begin{verbatim}
# useradd -m -g [pradine_grupe] -G [papildomos_grupes] -s
    [prisijungimo_kiautas] [vartotojo_vardas]
  \end{verbatim}

  \begin{itemize}
    \item \textbf{-m} argumentas pasako programai sukurti naują namų
      direktoriją vartotojui \textsl{/home/vartotojo$\_$vardas}. Savo
      direktorijoje vartotojas turi tiek rašymo, tiek skaitymo
      teises. Vartotojas savo direktorijoje gali net diegti
      programas. Taip pat tokioje direktorijoje yra saugomi visos
      vartotojų konfigūracinės bylos, taip vadinamos 'taškinės bylos' (
      jų pavadinimas prasideda tašku, kaip pavyzdžiui \textsl{.vimrc}
      ). Jos yra 'paslėptos'. Pagal numatytus nustatymus \textbf{ls}
      jų nerodo. Tam, kad jas pamatyti reikia prie komandos pridėti
      \textsl{-a} argumentą. Kuomet kyla konfliktas su vartotojo
      konfigūracinėmis bylomis ir globaliomis konfigūracinėmis bylomis
      - vartotojo konfigūracinės bylos lieka dominuojančioje
      pozicijoje. Dažniausios taškinės bylos, kurios yra keičiamos
      vartotojo namų direktorijoje yra \textsl{.xinitrc} ir
      \textsl{.bashrc}. Sekančios konfigūracinės bylos atsakingos už
      xinit ir Bash atitinkamai. Šios konfigūracinės bylos leidžia
      vartotojui keisti langų tvarkyklę, kuri pasirodo po prisijungimo
      lango, alternatyvius vardus, vartotojo sukurtas komandas, bei
      aplinkos kintamuosius atitinkamai. Kuomet yra sukuriamas naujas
      vartotojas, visos jo taškinės bylos, kaip konfigūracijų
      pagrindas, yra imamos iš \textsl{/etc/skel} direktorijos. 
    \item \textbf{-g} argumentas nusako prie kokios grupės priskirti
      vartotoją iškarto po prisijungimo prie sistemos. Grupės vardas
      turi egzistuoti. Jeigu vietoj grupės vardo yra nurodomas grupės
      numeris, jis turi nurodyti į egzistuojančią grupę. Jeigu
      parametras nėra nurodomas, useradd programos veiksmai priklausys
      nuo \textsl{USERGROUPS$\_$ENAB} kintamojo, kuris yra aprašytas
      \textsl{/etc/login.defs}.
    \item \textbf{-G} argumentas nusako prie kokių grupių priskirti
      vartotoją. Kiekviena grupė yra pateikiama eilute, grupės
      atskiriamos kablelių, tarpų būti negali. Pagal numatytus
      nustatymus, vartotojas priklauso tik tai grupei, kuri yra
      nusakyta \textsl{-g} argumentu.
    \item \textbf{-s} argumentas nusako absoliutų kelią ir pavadinimą
      numatyto prisijungimo kiauto. Arch Linux init skriptai naudoja
      Bash kiautą. Po sistemos krovimo, vartotojas bus perkeltas prie
      jo nurodyto kiauto. Verta įsitikinti, kad vartotojo nurodytas
      kiautas egzistuoja sistemoje.
  \end{itemize}

  Naudingos vartotojų grupės:

  \begin{itemize}
    \item \textbf{audio} - garso reikmėms
    \item \textbf{floppy} - diskelių reikmėms
    \item \textbf{lp} - spausdinimo reikmėms
    \item \textbf{optical} - lokalaus optinio diskinio įrenginio
      reikmėms
    \item \textbf{storage} - laikmenų reikmėms
    \item \textbf{video} - vaizdo, bei sistemos greitinimo reikmėms
    \item \textbf{wheel} - naudojant sudo
    \item \textbf{games} - esant būtinybei turėti rašymo teises
      žaidimams
    \item \textbf{power} - kompiuterio energijos reikmėms (
      išjungimas, perkrovimas )
    \item \textbf{scanner} - skaitytuvo reikmėms
  \end{itemize}

  Tipinio vartotojo sukūrimo komanda atrodytų taip:

  \begin{verbatim}
# useradd -m -g users \
    -G audio,lp,optical,storage,video,wheel,games,power,scanner \
    -s /bin/bash avartotojas
  \end{verbatim}

  Toliau, reikia sukurti slaptažodį naujam vartotojui, naudojant
  \textsl{passwd} komandą:

  \begin{verbatim}
# passwd avartotojas
  \end{verbatim}

  Dabar, naujas ne-root vartotojas jau yra pilnai sukurtas. Galite
  atsijungti iš root vartotojo su \textsl{logout} ir prisijungti kaip
  ne-root vartotojas.

    \subsection{Vartotojo pašalinimas}

      % TODO
    
      Jeigu nutinka kažkokia klaida, arba yra būtinybė pakeisti
      vartotojo prisijungimo vardą, arba dėl kitos priežastis, galima
      labai lengvai pašalinti sistemos vartotoją su \textsl{userdel}:
      
      \begin{verbatim}
# userdel -r [vartotojo-vardas]
      \end{verbatim}

      \begin{itemize}
        \item \textbf{-r} argumentas pasako programai pašalinti visas su
          vartotoju susietas bylas ir direktorijas: jo namų direktorija,
          su visomis bylomis, bei elektroninio pašto vietą.
      \end{itemize}

  \section{Sudo diegimas ir konfigūravimas ( nebūtina )}

    % TODO

    Sudo diegimas su pacman yra labai paprastas:
    
    \begin{verbatim}
# pacman -S sudo
    \end{verbatim}

    Norint pridėti ne-root vartotoją prie sudo vartotojų grupės,
    pirmiausiai reikia sukonfigūruoti sudo konfigūracinę bylą. Tai
    galima padaryti su \textsl{visudo} komanda, paleidžiant ją su root
    vartotojo teisėmis. 

    Pagal numatytus nustatymus, visudo komanda naudos \textsl{vi}
    teksto redaktorių. Jeigu nežinote kaip naudotis tokiu teksto
    redaktoriumi, patartina pasirinkti kitą. Kitą redaktorių galima
    nustatyti pakoreguojant \textsl{EDITOR} aplinkos kintamąjį (
    pavyzdyje naudosime nano ):

    \begin{verbatim}
# EDITOR=nano visudo
    \end{verbatim}

    \pagargraph{Pastaba} Reikia įsidėmėti, jog kintamojo deklaravimas
    bei visudo komandos paleidimas vykdomas vienu metu, per vieną
    eilutę. Jeigu komandas parašyti atskirai, t.y. per dvi eilutes -
    niekas nesuveiks. 

    Visodo komanda turi atidaryti \textsl{/etc/sudoers} konfigūracinę
    bylą.  Pastaroji komanda neredaguoja bylos tiesiogiai. Ji
    perkopijuoja konfigūracinę bylą į laikiną saugyklą ( tmp ) ir
    atidaro vartotojui redagavimo aplinką ( vi arba kitą redaktorių,
    kuris yra nustatomas per \textsl{EDITOR} kintamąjį ). Tuomet, kai
    redagavimas yra baigiamas - \textsl{/etc/sudoers} laikinoji byla
    yra sutikrinama ir, jeigu konfigūracinis formatas yra teisingas -
    perrašo pradinę bylą.

    \paragraph{Perspėjimas} Tiesioginis \textsl{/etc/sudoers} bylos
    redagavimas yra griežtai nerekomenduojamas. Formato klaidos gali
    privesti iki to, jog negalima bus niekaip pasiekti root vartotojo.

    Ankščiau mes pridėjome mūsų naujai sukurtą ne-root vartotoją prie
    ``wheel'' grupės. Dabar tereikia pasakyti sudo, kad visi
    vartotojai, kurie yra ``wheel'' grupėje gali operuoti sistemoje ir
    su root teisėmis, kuomet yra vykdoma \textsl{sudo} komanda. Tam
    tereikia atkomentuoti tokią eilutę:

    \begin{verbatim}
%wheel ALL=(ALL) ALL
    \end{verbatim}

    Dabar, jeigu kursite naujus vartotojus, jums tereikės juos pridėti
    prie ``wheel'' grupės ir jie galės naudotis root vartotojo teisėmis.
