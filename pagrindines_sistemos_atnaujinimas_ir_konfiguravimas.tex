\chapter{Pagrindinės sistemos atnaujinimas ir konfigūravimas}

  Šviežia sistema pradės krautis ir krovimas bus užbaigtas tuomet, kai
  pasirodys prisijungimo sąsaja ( prompt ).

  \textbf{Sveikiname, ir sveiki atvykę į naują Arch Linux pagrindinę
    sistemą!}

  Nauja Arch Linux sistema ir pilnavertė GNU/Linux aplinka, kuri yra
  paruošta konfigūravimui. Nuo šio žingsnio, Jūs esate pats savo
  sistemos kalvis. Jūsų sistema bus tokia, kokia Jums yra reikalinga.

  Prisijunkite kaip root vartotojas. Toliau tęsime su pacman ir
  sistemos atnaujinimu.

  \paragraph{Pastaba} Yra galimos virtualios konsolės. Jas galite
  pasiekti per Alt+F1 ... Alt+F6.

  \section{Žingsnis 1: Tinklo konfigūravimas}

    \textsl{Sekantis skyrius bus jums vadovo pozicijoje konfigūruojant
    tinklą, jeigu jis jums neveikia.}

    Jeigu įdiegimo metu teisingai buvo atliktas tinklo konfigūravimas,
    tinklas jums turėtų veikti. Tai galima patikrinti su ping komanda:

    \begin{verbatim}
ping -c www.google.lt
    \end{verbatim}

    \textsl{Jeigu tinklas yra pajungtas, tęskite su Žingnis 2.}

    Jeigu po ping bandymo ateina pranešimas apie ``unknown host'',
    tinklas nėra teisingai sukonfigūruotas. Pirmą, ką reiktų padaryti,
    tai dar kartą patikrinti konfigūracines bylas:

    \begin{itemize}
      \item \textbf{/etc/rc.conf} Ypač HOSTNAME= bei NETWORKING
        sekcijas.
      \item \textbf{/etc/hosts} Patikrinkite nustatymų formatą.
      \item \textbf{/etc/resolv.conf}, jeigu naudojate statinį IP
        adresą. Jeigu naudojate DHCP, byla yra sukuriama ir ištrinama
        dinamiškai. 
    \end{itemize}

    \subsection{LAN tinklas}

      Patikrinkite savo tinklo sąsajas su

      \begin{verbatim}
# ifconfig -a
      \end{verbatim}

      Įvykdžius šią komandą, bus pateiktos visos tinklo sąsajos,
      kurias sistema atpažino. Turėtumėt pamatyti kažką panašaus į
      \textsl{eth0} arba \textsl{eth1}.

      \begin{itemize}
        \item Statinis IP

          Jeigu yra būtina, galima nustatyti savo statinį IP:

          \begin{verbatim}
# ifconfig eth0 <ip adresas> netmask <netmask> up
          \end{verbatim}

          Ir numatytą gateway su:

          \begin{verbatim}
# route add default gw <gateway-ip-adresas>
          \end{verbatim}

          Patikrinkite ar \textsl{/etc/resolv.conf} byloje yra įrašyti
          DNS serveriai ir pridėkite juos, jeigu jų nėra. Patikrinkite
          savo tinklą su ping komanda ir jeigu tinklas atsirado -
          įveskite viską į savo \textsl{rc.conf} bylą, kaip aprašyta
          aukščiau, pirmame skyriuje.

        \item DHCP

          Jeigu naudojate DHCP, pabandykite tiesiog:

          \begin{verbatim}
# dhcpcd eth0
          \end{verbatim}

          Jeigu tai suveiks - įveskite reikiamus konfigūracijos
          parametrus į \textsl{/etc/rc.conf}, kaip nurodyta pirmoje dalyje.
      \end{itemize}

    \subsection{Belaidis tinklas}

      \begin{itemize}

        \item Patikrinkite ar reikiama tvarkyklė sukūrė naudojamą
          bevielio tinklo sąsają:

          \begin{verbatim}
# iwconfig
          \end{verbatim}

          Komanda turėtų parodyti esamas bevielio tinklo sąsajas. Jo
          pavadinimai gali būti \textsl{wlan0}, \textsl{wlan1} arba
          \textsl{eth1}.

        \item Įjunkite norimą sąsają su \textsl{ifconfig <sąsaja>},
          pavyzdžiui:

          \begin{verbatim}
# ifconfig wlan0 up
          \end{verbatim}

          \paragraph{Pastaba} Jeigu išmes klaidos pranešimą
          'SIOCSIFFLAGS: No such file or directory', tai reiškia, jog
          tinklo plokštei trūksta firmware'o.

        \item Susiekite norimą tinklą su sąsają. Sekantis procesas
          priklauso nuo tinklo tipo, prie kurio norima prisijungti,
          tačiau pagrindinis reikalavimas yra žinoti bevielio tinklo
          ESSID, t.y. tinklo pavadinimą ( pavyzdžiui 'eduroam' ):

          \begin{itemize}
            \item Pavyzdys, naudojant neužkoduotą tinklą:

              \begin{verbatim}
# iwconfig wlan0 essid "eduroam"
              \end{verbatim}

            \item Pavyzdys, naudojant WEP kodavimą ir šešioliktainį
              raktą:

              \begin{verbatim}
# iwconfig wlan0 essid "eduroam" key 0241baf34c
              \end{verbatim}

            \item Pavyzdys, naudojant WEP ir ASCII raktą:

              \begin{verbatim}
# iwconfig wlan0 essid "eduroam" key s:secret-pass
              \end{verbatim}

            \item Naudojant WPA, prisijungimas prie tinklo reikalauja
              truputi daugiau pastangų:

              \begin{verbatim}
# wpa_passphrase eduroam "slaptazodis" > /etc/wpa_supplicant.conf
# wpa_supplicant -B -Dwext -i wlan0 -c /etc/wpa_supplicant.conf
              \end{verbatim}

              \paragraph{Pastaba} Po -D esantis argumentas 'wext' yra
              tvarkyklės pavadinimas, kuris yra naudojamas sąsajai su
              bevieliu tinklu.

          \end{itemize}

          Patikrinkite ar tinklas teisingai sukonfigūruotas:

          \begin{verbatim}
# iwconfig wlan0
          \end{verbatim}

        \item Kreiptis į bevieli tinklą atliekama paprasčiausiai su
          \textsl{dhcpcd <sąsaja>}:
          
          \begin{verbatim}
# dhcpcd wlan0
          \end{verbatim}

        \item Patikrinkite ar tinklas sukonfigūruotas teisingai:

          \begin{verbatim}
# ping -c www.google.lt
          \end{verbatim}

          Jeigu viskas buvo atlikta teisingai - tinklas turėtų veikti.
          
      \end{itemize}

    \subsection{Proxy serveris}

    Jeigu esate už proxy serverio, pakoreguokite \textsl{/etc/wgetrc}
    įgalindami \textsl{http$\_$proxy}, bei \textsl{ftp$\_$proxy}. 

    \subsection{Analoginis modemas, ISDN ir DSL (PPPoE)}

      \subsubsection{Analoginis modemas}

      \subsubsection{ISDN}

      \subsubsection{DSL}

    % TODO

  \section{Žingsnis 2: Atnaujinimas su pacman}

    Dabar, kai turime veikiantį priėjimą prie tinklo, galime
    atnaujinti savo sistemą, pasitelkdami pacman.

    \subsection{Kas yra pacman ?}

    Pacman yra Arch Linux paketų tvarkyklė ( \textbf{pac}kage
    \textbf{man}ager ). Pacman yra parašytas su C kalba ir nuo
    pagrindų yra projektuotas kaip lengva, greita ir mažai darbinės
    atminties reikalaujanti programa. Pacman prižiūri kiekvieną
    paketą, kuris yra įdiegtas į sistemą: atlieka įdiegimą,
    pašalinimą, atnaujinimą, rankinių būdų sukompiliuotų paketų
    įdiegimą, priklausomybių tikrinimą, nuotolinę, bei vidinę paketų
    paiešką ir kita. Pacman programos pateikiama informacija yra
    suprantama ir suteikia ETA kiekvieno paketo atsiuntimo metu. Arch
    naudoja pkg.tar.gz archyvus ir šiuo metu yra vykdomi darbai
    pereinant prie pkg.tar.xz formato.

    Dabar pacman bus naudojamas atsisiunčiant paketus iš paketų
    saugyklos ir diegimui į sistemą.

    \subsection{Paketų saugyklos}

      Dabartiniu metu pacman siūlo tokias paketų saugyklos galimybes:

      \subsubsection{[core]} 

        Paprastu principu remiantis, [core] suteikia prieigą tik po
        vieną įrankį, skirtą atlikti tam tikrą pagrindinės Arch Linux
        sistemos darbą. GNU toolchan, sisteminis branduolys ( Linux
        Kernel ), vienas redaktorius, vienas komandinės eilutės
        klientas ir kt. ( tačiau yra ir keletas išimčių, kaip
        pavyzdžiui vi ir nano yra pasiekiami [core] saugykloje
        ). Saugykloje yra tie pagrindiniai paketai, kurie turi būti
        įdiegti į bet kokią veikiančią sistemą, norint užtikrinti, kad
        sistema veiktų stabiliai ir patikimai. Tai yra visiškai
        sistemiškai kritiniai paketai.

        \begin{itemize}
          \item Pagrindinės sistemos programuotojų palaikomi
          \item Visi paketai yra binariniai
          \item Prieinami per pacman
          \item \textsl{Core} sistemos diegimo atvaizdas susideda vien
            iš [core] saugykloje saugomų paketų.
        \end{itemize}

      \subsubsection{[extra]}

        Saugykloje [extra] saugomi visi paketai, kurie nėra būtini
        bazinėje Arch Linux sistemoje, tačiau jie suteikia pilną Arch
        Linux darbo aplinką, tad be jų neišsiversti. X, KDE, bei
        Apache yra saugomas sekančioje saugykloje.

        \begin{itemize}
          \item Pagrindinės sistemos programuotojų palaikomi
          \item Visi paketai yra binariniai
          \item Prieinami per pacman
        \end{itemize}
      
      \subsubsection{[testing]}

        Sekančia saugykla patartina naudotis tik labai patyrusiems
        vartotojams, todėl šiame vadove jos neapžvelgsime.
      
      \subsubsection{[community]}

        Saugykla yra prižiūrima Arch Linux patikimų vartotojų (
        \textsl{TU} ) ir paprasčiausiai yra binarinis AUR
        variantas. Jame saugomi binariniai paketai, kurie yra
        suorganizuoti PKGBUILD pagalba iš AUR, kurie sulaukė
        daugiausiai balsų ir buvo priimti į saugyklą. Kaip ir
        kiekviena saugykla, išvardinta aukščiau, [community] galima
        pasiekti pacman pagalba.

        \begin{itemize}
          \item TU palaikomi
          \item Visi paketai yra binariniai
          \item Prieinami per pacman
        \end{itemize}
      
      \subsubsection{[multilib]}

        Vartotojai, kurie naudoja 64 bitų architektūrą turi poreikį
        diegti paketus, kurie nėra pritaikyti tokiai
        architektūrai. Paprasti 32 bitų paketai gali būti naudojami 64
        bitų versijoje, tačiau sistemoje turi egzistuoti specifinės
        bibliotekos, kurios ir yra saugomos [multilib] saugykloje. 

        \begin{itemize}
          \item Pagrindinės sistemos programuotojų palaikomi
          \item Paketai yra binariniai
          \item Prieinami per pacman
        \end{itemize}

        \paragraph{Pastaba} Norint naudotis saugykla, reikia pridėti
        kelias eilutes į \textsl{/etc/pacman.conf}

        \begin{verbatim}
[multilib]
Include = /etc/pacman.d/mirrorlist
        \end{verbatim}

      \subsubsection{AUR (nepalaikomas)}

        AUR saugomi visi oficialiai nepalaikomi paketai, kurie negali
        būti pasiekiami per pacman. AUR'e nėra jokių binarinių
        paketų. Tuo tarpu, AUR suteikia daugiau nei šešiolika
        tūkstančių PKGBUILD skriptų, kurių pagalba galima labai
        lengvai sukompiliuoti bet kokį norimą paketą iš pradinio
        kodo. Kuomet AUR PKGBUILD skriptas surenka pakankamai balsų,
        jo binarinis variantas yra perkeliamas į [community]
        saugyklą. Tam taip pat turi pritarti ir \textsl{TU}.

        \begin{itemize}
          \item TU palaikomi
          \item Visi PKGBUILD yra skriptai
          \item Pagal numatytus nustatymus, neprieinamas per pacman
        \end{itemize}

        Sekančią saugyklą galima pasiekti kitais būdais,
        t.y. pasitelkiant pacman apgaubtu ( wrapper ).

    \subsection{/etc/pacman.conf}

      Kiekvieno pacman paleidimo metu, jis skaitys
      \textsl{/etc/pacman.conf}. Konfigūracinė byla yra suskaldyta į
      du blokus arba saugyklas. Kiekvienas blokas nusako paketų
      saugyklą, kurią pacman gali naudoti paketų paieškai. Išimtis yra
      opcijų blokas, kuris nustato globalius kintamuosius.

      Verta pastebėti, jog pagal numatytus nustatymus viskas turi
      veikti, taigi bylos redagavimas šiame žingsnyje nėra būtinas,
      tačiau patikrinimas yra rekomenduojamas. 

      \begin{verbatim}
# nano /etc/pacman.conf
      \end{verbatim}

      Įjunkite visas norimas saugyklas, pašalindami komentaro ($\#$)
      ženklą eilutės pradžioje.

      \paragraph{Pastaba} Įsitikinkite, jog atkomentavote tiek
      saugyklą, kurios pavadinimas apskliaustas laužtiniais
      skliaustais ([saugyklos-pavadinimas]), tiek ir \textsl{Include} eilutę, kuri seka
      iškarto po saugyklos pavadinimo. Tai dažna klaida.

    \subsection{/etc/pacman.d/mirrorlist}

      Konfigūracinė byla nusako saugyklų veidrodžius bei prioritetus.

    \subsection{Mirrorcheck sistemos naujumui tikrinimui}

    \subsection{Paketų ignoravimas}

    \subsection{Konfigūracinių bylų ignoravimas}

    \subsection{Supažindinimas su pacman}

    \section{Sistemos atnaujinimas}

      \subsection{Arch dabartinės versijos modelis}

