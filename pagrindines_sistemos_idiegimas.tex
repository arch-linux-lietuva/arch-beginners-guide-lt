\chapter{Pagrindinės sistemos įdiegimas}
  \section{Naujausios įdiegimo laikmenos gavimas}

  Arch Linux oficialią įdiegimo laikmeną galite gauti iš
  \textsl{http://archlinux.org/download}. Vadovo rašymo metu,
  naujausia versija yra 2010.05.
  \begin{itemize}
    \item Tiek \textsl{Core} tiek \textsl{Netinstall} atvaizdai
      suteikia tik pagrindinę sistemą. Verta pastebėti, jog
      pagrindinėje Arch Linux sistemoje nėra jokios grafinės
      aplinkos. Pagrindinė sistema susideda iš GNU įrankių grandinės ( kompiliatoriaus,
      asmeblerio, linkerio, kt. ), Linux branduolio ir kelių papildomų
      bibliotekų ir modulių.
    \item Įdiegimas yra palengvintas tiek \textsl{Core}, tiek
      \textsl{Netinstall} atvaizduose.
    \item \textsl{Netinstall} atvaizdas yra mažesnis, bet jame
      visiškai nėra pagrindinės sistemos paketų. Visa sistema yra
      parsiunčiama iš interneto.
  \end{itemize}

  \section{Sistemos įdiegimas iš egzistuojančios GNU/Linux
    distribucijos}

  Arch Linux yra pakankamai lankstus, kad galėtų būti įdiegtas iš
  kitos, egzistuojančios distribucijos į laisvą patriciją arba iš Live
  CD.
  Įdiegimą iš egzistuojančios GNU/Linux distribucijos apžvelgsime
  vadovo pabaigoje.

  % TODO:
  % http://wiki.archlinux.org/index.php/Install_from_Existing_Linux

  \section{Įdiegimas iš CD laikmenos}
  
  Iškepkite atsiųstą .iso atvaizdą į CD arba DVD su mėgstama CD/DVD
  rašymo programa ir tęskite savo kelią į kitą skyrių \textsl{Arch
    Linux įdiegimo krovimas}

  \section{Diegimas iš Flash atminties kortelės arba USB atmintinės}

  Sekantis metodas veiks bet kokiam Flash atminties tipui, kurį BIOS
  leis krauti paleidimo metu, būtų tai kortelių skaitytuvas arba USB
  portas.

  \paragraph{UNIX Metodas}
  
  Įdėkite tuščią arba nereikalingą flash laikmeną, nustatykite iki jos
  kelią ir įrašykite .iso atvaizdą, pasitelkus \textsl{/bin/dd}
  programą:

  \begin{verbatim}
dd if=archlinux-2010.05-{core|netinstall}-{i686|x86_64|dual}.iso\
  of=/dev/sdx
  \end{verbatim}

  kur $if=$ yra kelias iki atvaizdo failo ir $of=$ yra kelias iki Jūsų
  flash laikmenos. Įsitikinkite, kad naudojate \textsl{/dev/sdx}, o ne
  \textsl{/dev/sdx1}. Jums reikės flash atminties tiek, kad joje
  tilptų 381MB duomenų.

  \paragraph{Patikrinkite md5sum}

  Pasižymėkite koks buvo įrašų (blokų) skaičius, kai buvo rašoma į
  laikmeną. Tuomet galima patikrinti:

  \begin{verbatim}
dd if=/dev/sdx count=irasu_skaicius status=noxfer | md5sum
  \end{verbatim}

  Patikrinimo kodas turi sutapti su parsiųsto atvaizdo md5sum.

  %TODO: Microsoft Windows Method

  \section{Arch Linux įdiegimo krovimas}
  
  Įdėkite CD ar Flash laikmeną, perkraukit kompiuterį ir paleiskite
  sistema iš CD ar Flash laikmenos. Jums gali prireikti pakeisti
  krovimosi eiliškumą BIOS nustatymuose arba paspausti kažkokį
  mygtuką. Dažniausiai toks mygtukas būna DEL, F1, F2, F11 arba
  F12. Pabandykite paspausti vieną iš jų, kuomet BIOS yra POST ( Power
  On Self-Test ) režime.

  Pastaba: Atminties reikalavimai baziniam įdiegimui yra tokie:
  \begin{itemize}
    \item \textsl{Core} : 128 MB RAM x86$\_$64/i686 ( pažymėti visi
      paketai, su swap particija )
    \item \textsl{Netinstall} : 128 MB RAM x86$\_$64/i686 ( pažymėti
      visi paketai, su swap particija )
  \end{itemize}

  Šiame žingsnyje turėtų pasirodyti pagrindinis menu. Pasirinkite
  Jums reikalingą opciją klaviatūros navigaciniais mygtukais ir padarę
  pasirinkimą paspauskite 'Enter' mygtuką.

  Dažniausiai, pirmą kartą kraunant Arch Linux \textsl{Boot Archlive}
  yra tas pasirinkimas, kurio Jums reikia. Tačiau, jeigu turite bėdų
  su libata/PATA arba neturite SATA (Serial ATA), pasirinkite
  \textsl{Boot Archlive [legacy IDE]}.

  Norint pakeisti GRUB pasirinkimus, paspauskite raidę
  \textbf{e}. Dauguma vartotojų norės pakeisti framebuffer
  rezoliuciją, patogesniam darbui. Pridėkite:
  \begin{verbatim}
vga=773
  \end{verbatim}
  kernelio eilutėje. Tuomet paspauskite \textbf{Enter} mygtuką, kad
  įgalinti 1024x768 framebuffer rezoliuciją. Kuomet viskas bus
  padaryta, paspauskite \textbf{b} mygtuką, kad pradėti sistemos
  krovimą.

  Dabar sistema turi pradėti krautis. Kuomet sistema pilnai pasikraus,
  turėtų pasirodyti prisijungimo galimybė. Prisijunkite kaip
  \textsl{root} vartotojas, kadangi Archlive diske iš \textsl{root}
  vartotojo nereikalaujama slaptažodžio.

  \paragraph{Klaviatūros išdėstymo keitimas}

  Jeigu turite ne US klaviatūros išdėstymą, tuomet Jums reikia ją
  pasikeisti. Tai galima atlikti su \textsl{km} komanda:
  \begin{verbatim}
km
  \end{verbatim}
  arba panaudoti \textsl{loadkeys} komandą:
  \begin{verbatim}
loadkeys layout
  \end{verbatim}
  ( pakeiskite \textsl{layout} su Jūsų pasirinktu klaviatūros
  išdėstymu ).

  Dauguma lietuviškų klaviatūrų yra US standarto su papildomais
  lietuviškais simboliais, kurie pasirodo vietoj skaičių eilės,
  esančios virš \textsl{q,w,e,r,t,y,u,i,o,p} klavišų eilės. Tai
  reiškia, kad daugumoje atveju, Jums visiškai nereikia keisti
  klaviatūros išdėstymo.

  \paragraph{Dokumentacija}

  Oficialus įdiegimo vadovas yra pasiekiamas iškarto diske. Norint jį
  pasiekti, persijunkite į kitą konsolę ( ALT+F2 ) ir tuomet surinkite
  tokią komandą:
  \begin{verbatim}
less /usr/share/aif/docs/official_installlation_guide_en
  \end{verbatim}

  \textsl{Less} komanda leis Jums peržvelgti visą vadovą
  puslapiais. Norint persijungti atgal į įdiegimą, tiesiog paspauskite
  Alt+F1 ir grįšite į pirmą konsolę, kurioje yra vykdomas įdiegimas.

  Prireikus vėl paskaityti dokumentaciją, tiesiog persijunkite į antrą
  terminalą su Alt+F2, norint grįžti prie įdiegimo - Alt+F1.

  Pastaba: Įsidėmėkite, kad oficialus įdiegimo vadovas apžvelgia tik
  pagrindinės sistemos įdiegimą ir konfigūravimą. Kai tik su
  pagrindine sistema yra susitvarkyta, rekomenduojama grįžti prie
  detalesnio vadovo, kuriame yra aprašyti visi reikalingi žingsniai po
  įdiegimo.

  \section{Įdiegimo paleidimas}

  Kuomet esate prisijungė kaip \textsl{root} vartotojas, pirmame
  terminale galite paleisti įdiegimo skriptą:
  \begin{verbatim}
/arch/setup
  \end{verbatim}

  \subsection{Įdiegimo šaltinio pasirinkimas}

  Po pasisveikinimo, Jūsų paprašys pasirinkti įdiegimo
  šaltinį. Priklausomai nuo ankstesnės laikmenos pasirinkimo,
  atitinkamai pasirinkite ir įdiegimo šaltinį.
  \begin{itemize}
    \item Jeigu pasirinkote \textsl{Core} įdiegimo atvaizdą, tęskite
      toliau prie skyriaus \textsl{Laikrodžio nustatymas}
    \item Jeigu pasirinkote \textsl{Netinstall}, Jums reiks rankiniu
      būdu užkrauti tinklo plokštės tvarkykles ( žinoma, jeigu sistema
      automatiškai neaptiks Jūsų turimos įrangos ). Udev yra labai
      naudingas įrankis, norint sužinoti Jūsų turimą įrangą. Tai
      galima patikrinti, pasitelkus \textsl{ifconfig -a} komandą.
  \end{itemize}

  \paragraph{Tinklo konfigūravimas (Netinstall)}

  Šitam žingsnyje, sistema turi parodyti jos rastus tinklo
  sąsajas. Jeigu sąsaja ir HWaddr ( HardWare address ) yra sąraše,
  tuomet Jūsų tinklo plokštė buvo sėkmingai aptikta ir jos tvarkyklės
  sėkmingai įkrautos į branduolį. Jeigu Jūsų tinklo plokštė atpažinta
  nebuvo, tuomet Jums reikia rankiniu būdu kelti tvarkykles į
  branduolį kitoje konsolėje.

  Atsiradus panašiam langui, bus paprašyta pasirinkti \textsl{sąsają},
  \textsl{zondą} arba \textsl{Atšaukti}. Pasirinkite reikiamą
  \textsl{sąsają} ir tęskite.

  Paskui įdiegimo programa paklaus Jūsų ar Jūs norite naudoti
  \textsl{DHCP}. Pasirinkus \textsl{Yes}, įdiegimo programa paleis
  \textbf{dhcpcd}, kuri aptiks tinklo sąsaja ir iš sąsajos reikalaus
  IP adreso. Jeigu pasirinksite \textsl{No}, tuomet įdiegimo programa
  paprašys Jūsų įvesti statišką IP, tinklo kaukę, transliatorių,
  tinklo sąsajos DNS IP, HTTP ir FTP proxy. Paskutiniame žingsnyje
  sistema parodys gautus nustatymus. Šitame žingsnyje galite patikrinti
  ar visi nustatymai yra teisingi.

  \paragraph{Greitasis (A)DSL paleidimas}

  ( Jeigu turite modemą arba routerį tilto režime, jungiantis per ISP
  jungtį )

  Persijunkite į kitą konsolę ( ALT+F2 ), prisijunkite kaip
  \textsl{root} vartotojas ir rašykite:

  \begin{verbatim}
pppoe-setup
  \end{verbatim}

  Jeigu viskas sukonfigūruota teisingai, pabaigoje galite prisijungti
  prie savo ISP su:

  \begin{verbatim}
pppoe-start
  \end{verbatim}

  Grįžkite į pirmą konsolę ( ALT+F1 ) ir tęskite įdiegimą
  \textsl{Laikrodžio nustatymas} skyriuje.

  \paragraph{Greitasis bevielio tinklo paleidimas}

  ( Jeigu turite bevielio ryšio galimybę įdiegimo metu )

  Šiuo metu bevielio ryšio tvarkyklės yra pasiekiamos ir iš įdiegimo
  aplinkos atvaizdo. Geras savo bevielio įrangos žinojimas šiuo atveju
  Jums labai padės. Įsidėmėkite, jog sekantys nustatymai galios tik
  dabartinėje aplinkoje. Kuomet baigsite įdiegimą, visus nustatymus
  reiks pakartoti įdiegtoje sistemoje, norint ir toliau naudotis
  bevieliu ryšiu.

  Taip pat verta pastebėti, jog šie žingsniai nėra privalomi, jeigu
  įdiegimas yra įmanomas ir be bevielio tinklo. Bevielio tinklo
  konfigūravimas gali būti atliktas ir po įdiegimo pabaigos, jau
  įdiegtoje sistemoje.

  Pagrindiniai žingsniai būtų tokie:

  \begin{itemize}
    \item Persijunkite į laisvą konsolę ( pavyzdžiui 3, ALT+F3 );
    \item Prisijunkite kaip \textsl{root} vartotojas;
    \item ( Nebūtina ) Identifikuokite savo bevielio ryšio sąsają:
      \begin{verbatim}
lspci | grep -i net
      \end{verbatim}
    \item Įsitikinkite, kad \textsl{udev} atpažino bevielio ryšio
      kortą ir pakrovė į branduolį reikalingus modulius. Tai galima
      patikrinti su \textsl{/usr/bin/iwconfig} komanda:
\begin{verbatim}
iwconfig
\end{verbatim}
      Rezultatas turėtų būti kažkas panašaus:
\begin{verbatim}
lo no wireless extensions.
eth0 no wireless extensions.
wlan0    unassociated  ESSID:""
         Mode:Managed  Channel=0  Access Point: Not-Associated   
         Bit Rate:0 kb/s   Tx-Power=20 dBm   Sensitivity=8/0  
         Retry limit:7   RTS thr:off   Fragment thr:off
         Power Management:off
         Link Quality:0  Signal level:0  Noise level:0
         Rx invalid nwid:0  Rx invalid crypt:0  Rx invalid frag:0
         Tx excessive retries:0  Invalid misc:0   Missed beacon:0
\end{verbatim}
      Šiame pavyzdyje, \textsl{wlan0} ir bus bevielio ryšio sąsają,
      per kurią galima jungtis prie bevielio tinklo.
    \item Sekantis žingsnis būtų įgalinti sąsają su
      \textsl{/sbin/ifconfig <sąsaja> up} komanda.
      Pavyzdys, turint \textsl{wlan0} sąsają:
\begin{verbatim}
ifconfig wlan0 up
\end{verbatim}
      Įsidėmėkite, jog Jūsų sąsajos pavadinimas gali būti kitoks, negu
      \textsl{wlan0}. Priklausomai nuo modulio, kurį užkrovė
      \textsl{udev}, Jūsų sąsajos pavadinimas gali būti
      \textsl{wlan0}, \textsl{eth1}, kt.

    \item Jeigu nežinote tiksliai savo bevielio tinklo \textsl{essid}
      arba norite pasijungti prie Jums nežinomo bevielio tinklo,
      naudokite \textsl{/sbin/iwlist <sąsaja> scan} komanda. 
\begin{verbatim}
iwlist wlan0 scan
\end{verbatim}
      Šį komanda peržvelgs šiuo metu pasiekiamus bevielio ryšio
      tinklus.
    \item Sekantis žingsnis būtų nurodyti Jūsų bevielio tinklo sąsajai
      tinklo \textsl{essid}. Priklausomai nuo bevielio tinklo kodavimo
      algoritmo (WEP, WPA arba išvis jokio), nurodymo procedūra gali
      skirtis. Pavyzdžiui galima padaryti prielaidą, jog Jūs norite
      prisijungti prie \textsl{}
      
  \end{itemize}
