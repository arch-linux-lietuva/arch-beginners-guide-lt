\chapter{Įvadas}

  \section{Įžanga}

  Sveikas skaitytojau. 

  Šita knyga padės Jums susipažinti, įdiegti ir susikonfigūruoti
  Arch Linux sistema Jūsų kompiuteryje. Arch Linux yra lengva, mažai
  vietos užimanti GNU/Linux distribucija, orientuota į kompetetingus
  vartotojus. Šitas vadovas yra orientuotas į naujus Arch vartotojus,
  tačiau tinka ir patyrusiam vartotojui, kaip informacijos šaltinis.

  \section{Svarbiausi Arch Linux distribucijos bruožai}
  \begin{itemize}
    \item Paprasta filosofija ir paprastas dizainas;
    \item Visi paketai kompiliuojami i686 ir x86$\_$64 architektūroms;
    \item BSD stiliaus paleidimo skriptai, kurie valdomi iš vieno, centrinio failo;
    \item mkinitcpio: Paprastas ir dinamiškas initramfs kūrėjas;
    \item Pacman paketų tvarkyklė yra lengva ir lanksti, reikalaujanti mažų sistemos resursų;
    \item Arch Build System yra paketų kūrimo sistema, paremta portų
      architektūra, kuri suteikia paprastą kiautą kuriant įdiegiamus Arch
      paketus iš pradinio kodo;
    \item Arch User Repository siūlo daugiau nei tūkstantį vartotojų sukurtų
      kūrimo skriptų ir galimybę patalpinti savo sukurtą kūrimo
      skriptą.
  \end{itemize}

  \section{Licenzija}

  Arch Linux, pacman, dokumentacijos ir skriptų autorinės teisės
  2002-2007 metais priklauso Judd Vinet, o nuo 2007 priklauso Aaron
  Griffin. Autorines teises saugo GNU General Public Licenzijos 2
  versija.

  \section{Arch problemų sprendimo būdas}
  
  Visos Arch distribucijos architektūra yra paremta vienu principu -
  viską išlaikyti kuo paprastesniam variante.

  Paprastumas šiame kontekste reiškia - be jokių bereikalingų
  pridėjimų, modifikacijų ar kompiliavimų. O trumpai - elegantiškas
  ir minimalistinis problemos sprendimo būdas.

  Kelios mintys apie paprastumo supratimą:
  \begin{itemize}
    \item ``Techniniu požiūriu 'paprastumas' nėra stabili
      pozicija. Geriau būti techniškai elegantišku su pritaikomomis
      žiniomis, negu būti lengvu naudojime ir technišku.'' - Aaron
      Griffin
    \item ``Subjektai neturi būti manipuliuojami, jeigu tai nėra
      būtina'' - Occam skustuvai. Skustuvas šiuo atveju vaidina
      nereikalingų operacijų pašalinimą, leidžiantį tęsti priėjimą prie
      paprasto paaiškinimo, metodo ar teorijos.
    \item ``Neįprasta mano technikos dalis remiasi
      paprastumu.. Progreso dydis visą laiką remiasi paprastumo principu.'' - Bruce Lee
  \end{itemize}

  \section{Apie šitą vadovą}

    Arch wiki yra labai geras informacijos šaltinis, tad pirmiausiai,
    kreipdamiesi pagalbos, įsitikinkite, kad Jūsų problema nėra
    aprašyta wiki puslapyje. Jeigu atsakimo į savo problemą taip ir neradote, galite
    kreiptis į $\#$archlinux IRC kanalą freenode serveryje arba galite
    kreiptis į forumą - http://bbs.archlinux.org; Taip pat galite
    kreiptis į Arch Linux Lietuva bendruomenę http://sls.archlinux.lt.

    Visas vadovas yra suskirstytas į keturias dalis:
    \begin{itemize}
      \item Dalis 1: Pagrindinės sistemos įdiegimas;
      \item Dalis 2: Arch Linux pagrindinės sistemos atnaujinimas ir
        konfigūravimas;
      \item Dalis 3: X serverio įdiegimas ir ALSA konfigūravimas;
      \item Dalis 4: Darbastalio aplinkos įdiegimas;
    \end{itemize}