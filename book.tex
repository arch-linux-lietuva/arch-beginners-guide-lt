\documentclass[a4paper]{book}
\usepackage[utf8x]{inputenc}
\usepackage[lithuanian]{babel}
\usepackage[L7x]{fontenc}
\begin{document}

\chapter{Įvadas}

  \section{Įžanga}

  Sveikas skaitytojau. 

  Šita knyga padės Jums susipažinti, įdiegti ir susikonfigūruoti
  Arch Linux sistema Jūsų kompiuteryje. Arch Linux yra lengva, mažai
  vietos užimanti GNU/Linux distribucija, orientuota į kompetetingus
  vartotojus. Šitas vadovas yra orientuotas į naujus Arch vartotojus,
  tačiau tinka ir patyrusiam vartotojui, kaip informacijos šaltinis.

  \section{Svarbiausi Arch Linux distribucijos bruožai}
  \begin{itemize}
    \item Paprasta filosofija ir paprastas dizainas;
    \item Visi paketai kompiliuojami i686 ir x86$\_$64 architektūroms;
    \item BSD stiliaus paleidimo skriptai, kurie valdomi iš vieno, centrinio failo;
    \item mkinitcpio: Paprastas ir dinamiškas initramfs kūrėjas;
    \item Pacman paketų tvarkyklė yra lengva ir lanksti, reikalaujanti mažų sistemos resursų;
    \item Arch Build System yra paketų kūrimo sistema, paremta portų
      architektūra, kuri suteikia paprastą kiautą kuriant įdiegiamus Arch
      paketus iš pradinio kodo;
    \item Arch User Repository siūlo daugiau nei tūkstantį vartotojų sukurtų
      kūrimo skriptų ir galimybę patalpinti savo sukurtą kūrimo
      skriptą.
  \end{itemize}

  \section{Licenzija}

  Arch Linux, pacman, dokumentacijos ir skriptų autorinės teisės
  2002-2007 metais priklauso Judd Vinet, o nuo 2007 priklauso Aaron
  Griffin. Autorines teises saugo GNU General Public Licenzijos 2
  versija.

  \section{Arch problemų sprendimo būdas}
  
  Visos Arch distribucijos architektūra yra paremta vienu principu -
  viską išlaikyti kuo paprastesniam variante.

  Paprastumas šiame kontekste reiškia - be jokių bereikalingų
  pridėjimų, modifikacijų ar kompiliavimų. O trumpai - elegantiškas
  ir minimalistinis problemos sprendimo būdas.

  Kelios mintys apie paprastumo supratimą:
  \begin{itemize}
    \item ``Techniniu požiūriu 'paprastumas' nėra stabili
      pozicija. Geriau būti techniškai elegantišku su pritaikomomis
      žiniomis, negu būti lengvu naudojime ir technišku.'' - Aaron
      Griffin
    \item ``Subjektai neturi būti manipuliuojami, jeigu tai nėra
      būtina'' - Occam skustuvai. Skustuvas šiuo atveju vaidina
      nereikalingų operacijų pašalinimą, leidžiantį tęsti priėjimą prie
      paprasto paaiškinimo, metodo ar teorijos.
    \item ``Neįprasta mano technikos dalis remiasi
      paprastumu.. Progreso dydis visą laiką remiasi paprastumo principu.'' - Bruce Lee
  \end{itemize}

  \section{Apie šitą vadovą}

    Arch wiki yra labai geras informacijos šaltinis, tad pirmiausiai,
    kreipdamiesi pagalbos, įsitikinkite, kad Jūsų problema nėra
    aprašyta wiki puslapyje. Jeigu atsakimo į savo problemą taip ir neradote, galite
    kreiptis į $\#$archlinux IRC kanalą freenode serveryje arba galite
    kreiptis į forumą - http://bbs.archlinux.org; Taip pat galite
    kreiptis į Arch Linux Lietuva bendruomenę http://sls.archlinux.lt.

    Visas vadovas yra suskirstytas į keturias dalis:
    \begin{itemize}
      \item Dalis 1: Pagrindinės sistemos įdiegimas;
      \item Dalis 2: Arch Linux pagrindinės sistemos atnaujinimas ir
        konfigūravimas;
      \item Dalis 3: X serverio įdiegimas ir ALSA konfigūravimas;
      \item Dalis 4: Darbastalio aplinkos įdiegimas;
    \end{itemize}

\chapter{Pagrindinės sistemos įdiegimas}
  \section{Naujausios įdiegimo laikmenos gavimas}

  Arch Linux oficialią įdiegimo laikmeną galite gauti iš
  \textsl{http://archlinux.org/download}. Vadovo rašymo metu,
  naujausia versija yra 2010.05.
  \begin{itemize}
    \item Tiek \textsl{Core} tiek \textsl{Netinstall} atvaizdai
      suteikia tik pagrindinę sistemą. Verta pastebėti, jog
      pagrindinėje Arch Linux sistemoje nėra jokios grafinės
      aplinkos. Pagrindinė sistema susideda iš GNU įrankių grandinės ( kompiliatoriaus,
      asmeblerio, linkerio, kt. ), Linux branduolio ir kelių papildomų
      bibliotekų ir modulių.
    \item Įdiegimas yra palengvintas tiek \textsl{Core}, tiek
      \textsl{Netinstall} atvaizduose.
    \item \textsl{Netinstall} atvaizdas yra mažesnis, bet jame
      visiškai nėra pagrindinės sistemos paketų. Visa sistema yra
      parsiunčiama iš interneto.
  \end{itemize}

  \section{Sistemos įdiegimas iš egzistuojančios GNU/Linux
    distribucijos}

  Arch Linux yra pakankamai lankstus, kad galėtų būti įdiegtas iš
  kitos, egzistuojančios distribucijos į laisvą patriciją arba iš Live
  CD.
  Įdiegimą iš egzistuojančios GNU/Linux distribucijos apžvelgsime
  vadovo pabaigoje.

  % TODO:
  % http://wiki.archlinux.org/index.php/Install_from_Existing_Linux

  \section{Įdiegimas iš CD laikmenos}
  
  Iškepkite atsiųstą .iso atvaizdą į CD arba DVD su mėgstama CD/DVD
  rašymo programa ir tęskite savo kelią į kitą skyrių \textsl{Arch
    Linux įdiegimo krovimas}

  \section{Diegimas iš Flash atminties kortelės arba USB atmintinės}

  Sekantis metodas veiks bet kokiam Flash atminties tipui, kurį BIOS
  leis krauti paleidimo metu, būtų tai kortelių skaitytuvas arba USB
  portas.

  \paragraph{UNIX Metodas}
  
  Įdėkite tuščią arba nereikalingą flash laikmeną, nustatykite iki jos
  kelią ir įrašykite .iso atvaizdą, pasitelkus \textsl{/bin/dd}
  programą:

  \begin{verbatim}
    dd if=archlinux-2010.05-{core|netinstall}-{i686|x86_64|dual}.iso of=/dev/sdx
  \end{verbatim}

  kur $if=$ yra kelias iki atvaizdo failo ir $of=$ yra kelias iki Jūsų
  flash laikmenos. Įsitikinkite, kad naudojate \textsl{/dev/sdx}, o ne
  \textsl{/dev/sdx1}. Jums reikės flash atminties tiek, kad joje
  tilptų 381MB duomenų.

  \paragraph{Patikrinkite md5sum}

  Pasižymėkite koks buvo įrašų (blokų) skaičius, kai buvo rašoma į
  laikmeną. Tuomet galima patikrinti:

  \begin{verbatim}
dd if=/dev/sdx count=irasu_skaicius status=noxfer | md5sum
  \end{verbatim}

  Patikrinimo kodas turi sutapti su parsiųsto atvaizdo md5sum.

  %TODO: Microsoft Windows Method

  \section{Arch Linux įdiegimo krovimas}
  
  Įdėkite CD ar Flash laikmeną, perkraukit kompiuterį ir paleiskite
  sistema iš CD ar Flash laikmenos. Jums gali prireikti pakeisti
  krovimosi eiliškumą BIOS nustatymuose arba paspausti kažkokį
  mygtuką. Dažniausiai toks mygtukas būna DEL, F1, F2, F11 arba
  F12. Pabandykite paspausti vieną iš jų, kuomet BIOS yra POST ( Power
  On Self-Test ) režime.

  Pastaba: Atminties reikalavimai baziniam įdiegimui yra tokie:
  \begin{itemize}
    \item \textsl{Core} : 128 MB RAM x86$\_$64/i686 ( pažymėti visi
      paketai, su swap particija )
    \item \textsl{Netinstall} : 128 MB RAM x86$\_$64/i686 ( pažymėti
      visi paketai, su swap particija )
  \end{itemize}

  Šiame žingsnyje turėtų pasirodyti pagrindinis menu. Pasirinkite
  Jums reikalingą opciją klaviatūros navigaciniais mygtukais ir padarę
  pasirinkimą paspauskite 'Enter' mygtuką.

  Dažniausiai, pirmą kartą kraunant Arch Linux \textsl{Boot Archlive}
  yra tas pasirinkimas, kurio Jums reikia. Tačiau, jeigu turite bėdų
  su libata/PATA arba neturite SATA (Serial ATA), pasirinkite
  \textsl{Boot Archlive [legacy IDE]}.

  Norint pakeisti GRUB pasirinkimus, paspauskite raidę
  \textbf{e}. Dauguma vartotojų norės pakeisti framebuffer
  rezoliuciją, patogesniam darbui. Pridėkite:
  \begin{verbatim}
vga=773
  \end{verbatim}
  kernelio eilutėje. Tuomet paspauskite \textbf{Enter} mygtuką, kad
  įgalinti 1024x768 framebuffer rezoliuciją. Kuomet viskas bus
  padaryta, paspauskite \textbf{b} mygtuką, kad pradėti sistemos
  krovimą.

  Dabar sistema turi pradėti krautis. Kuomet sistema pilnai pasikraus,
  turėtų pasirodyti prisijungimo galimybė. Prisijunkite kaip
  \textsl{root} vartotojas, kadangi Archlive diske iš \textsl{root}
  vartotojo nereikalaujama slaptažodžio.

  \paragraph{Klaviatūros išdėstymo keitimas}

  Jeigu turite ne US klaviatūros išdėstymą, tuomet Jums reikia ją
  pasikeisti. Tai galima atlikti su \textsl{km} komanda:
  \begin{verbatim}
km
  \end{verbatim}
  arba panaudoti \textsl{loadkeys} komandą:
  \begin{verbatim}
loadkeys layout
  \end{verbatim}
  ( pakeiskite \textsl{layout} su Jūsų pasirinktu klaviatūros
  išdėstymu ).

  Dauguma lietuviškų klaviatūrų yra US standarto su papildomais
  lietuviškais simboliais, kurie pasirodo vietoj skaičių eilės,
  esančios virš \textsl{q,w,e,r,t,y,u,i,o,p} klavišų eilės. Tai
  reiškia, kad daugumoje atveju, Jums visiškai nereikia keisti
  klaviatūros išdėstymo.

  \paragraph{Dokumentacija}

  Oficialus įdiegimo vadovas yra pasiekiamas iškarto diske. Norint jį
  pasiekti, persijunkite į kitą konsolę ( ALT+F2 ) ir tuomet surinkite
  tokią komandą:
  \begin{verbatim}
less /usr/share/aif/docs/official_installlation_guide_en
  \end{verbatim}

  \textsl{Less} komanda leis Jums peržvelgti visą vadovą
  puslapiais. Norint persijungti atgal į įdiegimą, tiesiog paspauskite
  Alt+F1 ir grįšite į pirmą konsolę, kurioje yra vykdomas įdiegimas.

  Prireikus vėl paskaityti dokumentaciją, tiesiog persijunkite į antrą
  terminalą su Alt+F2, norint grįžti prie įdiegimo - Alt+F1.

  Pastaba: Įsidėmėkite, kad oficialus įdiegimo vadovas apžvelgia tik
  pagrindinės sistemos įdiegimą ir konfigūravimą. Kai tik su
  pagrindine sistema yra susitvarkyta, rekomenduojama grįžti prie
  detalesnio vadovo, kuriame yra aprašyti visi reikalingi žingsniai po
  įdiegimo.

  \section{Įdiegimo paleidimas}

  Kuomet esate prisijungė kaip \textsl{root} vartotojas, pirmame
  terminale galite paleisti įdiegimo skriptą:
  \begin{verbatim}
    /arch/setup
  \end{verbatim}

  \subsection{Įdiegimo šaltinio pasirinkimas}

  Po pasisveikinimo, Jūsų paprašys pasirinkti įdiegimo
  šaltinį. Priklausomai nuo ankstesnės laikmenos pasirinkimo,
  atitinkamai pasirinkite ir įdiegimo šaltinį.
  \begin{itemize}
    \item Jeigu pasirinkote \textsl{Core} įdiegimo atvaizdą, tęskite
      toliau prie skyriaus \textsl{Laikrodžio nustatymas}
    \item Jeigu pasirinkote \textsl{Netinstall}, Jums reiks rankiniu
      būdu užkrauti tinklo plokštės tvarkykles ( žinoma, jeigu sistema
      automatiškai neaptiks Jūsų turimos įrangos ). Udev yra labai
      naudingas įrankis, norint sužinoti Jūsų turimą įrangą. Tai
      galima patikrinti, pasitelkus \textsl{ifconfig -a} komandą.
  \end{itemize}

  \paragraph{Tinklo konfigūravimas (Netinstall)}

  Šitam žingsnyje, sistema turi parodyti jos rastus tinklo
  sąsajas. Jeigu sąsaja ir HWaddr ( HardWare address ) yra sąraše,
  tuomet Jūsų tinklo plokštė buvo sėkmingai aptikta ir jos tvarkyklės
  sėkmingai įkrautos į branduolį. Jeigu Jūsų tinklo plokštė atpažinta
  nebuvo, tuomet Jums reikia rankiniu būdu kelti tvarkykles į
  branduolį kitoje konsolėje.

  Panašaus ekrano... % TODO

\end{document}
